\documentclass{article}
\usepackage[utf8]{inputenc}
\usepackage{geometry}
\usepackage{array}
\usepackage{float}
\usepackage{enumitem}
\usepackage{layout}
\geometry{margin=1in}
\usepackage{graphicx}
\usepackage{afterpage}
\usepackage{fancyhdr}
\usepackage{float}
\usepackage{tabularx}
\usepackage{graphicx}


\pagestyle{fancy}
\fancyhf{} 
\lhead{\textit{Bases de donnees}} 
\rhead{\thepage}  

\title{Ingénierie des panneaux solaires}
\author{}
\date{\vspace{-5ex}}
\geometry{margin=1in}
\begin{document}



\maketitle

\begin{center}
    \begin{tabular}{|l|l|}
        \hline
        \multicolumn{2}{|c|}{\textbf{Notions Systèmes - Analyse et conception de systèmes}} \\
        \hline
        \textbf{Version du document} & 1.1 \\
        \hline
        \textbf{Professeur} & Illyas Harti\\
        \hline
        \textbf{Étudiant} & Sami Boufassa\\
                            
        \hline
        \textbf{Date} & 11 Novembre 2024 \\
        \hline
    \end{tabular}
\end{center}

\tableofcontents

\newpage

\section{Introduction}


\subsection{Système cible}


\subsection{Angles d'Analyse}

\subsection{}


\section{Identification du périmètre systèmique}


\subsection{Environement et systèmes externes}
\subsection{Parties prenantes}
\subsection{Diagramme de contexte} 



\section{Analyse et modélisation opérationnelle}
\subsection{Expression des besoins}
\subsection{Diagramme de cycle de vie}

\subsection{Diagramme de scénario opérationnel}


\section{Analyse et modélisation fonctionnelle}



\subsection{Exigences fonctionnelles}

\clearpage


\subsection{Diagramme de scénario fonctionnel}

\clearpage


\subsection{}

\clearpage


\section{Analyse et modélisation organique}



\subsection{Exigences techniques}

\clearpage


\subsection{Diagramme de scénario organique}

\clearpage


\subsection{}

\clearpage


\

\section{Annexes}


\end{document}