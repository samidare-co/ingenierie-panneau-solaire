\documentclass{article}
\usepackage[utf8]{inputenc}
\usepackage{geometry}
\usepackage{array}
\usepackage{float}
\usepackage{enumitem}
\usepackage{layout}
\geometry{margin=1in}
\usepackage{graphicx}
\usepackage{afterpage}
\usepackage{fancyhdr}
\usepackage{float}
\usepackage{tabularx}
\usepackage{graphicx}


\pagestyle{fancy}
\fancyhf{} 
\lhead{\textit{Bases de donnees}} 
\rhead{\thepage}  

\title{Ingénierie des panneaux solaires}
\author{}
\date{\vspace{-5ex}}
\geometry{margin=1in}
\begin{document}



\maketitle

\begin{center}
    \begin{tabular}{|l|l|}
        \hline
        \multicolumn{2}{|c|}{\textbf{Notions Systèmes - Analyse et conception de systèmes}} \\
        \hline
        \textbf{Version du document} & 1.1 \\
        \hline
        \textbf{Professeur} & Illyas Harti\\
        \hline
        \textbf{Étudiant} & Sami Boufassa\\
                            
        \hline
        \textbf{Date} & 11 Novembre 2024 \\
        \hline
    \end{tabular}
\end{center}

\tableofcontents

\newpage

\section{Introduction}
L'énergie solaire est une source d'énergie renouvelable de plus en plus utilisée à l'échelle mondiale. Les systèmes photovoltaïques (PV), composés principalement de panneaux solaires, jouent un rôle crucial dans la transition énergétique.
Le système d'interet est un système photovoltaique intelligent. Les systemes photovoltaiques sont utilises sous plusieurs formes telles que les centrales solaires, les chauffe-eau solaires, les lampadaires solaires, les panneaux solaires, etc. et dans differents contextes tels que les habitations, les entreprises, les usines, les écoles, les hôpitaux, etc.  
Nous retrouvons bien la toutes les caractéristiques d'un système complexe : 
\begin{itemize}
    \item Un contexte : environnement dans lequel le système s'intègre en l'occurence un reseau électrique. 
    \item Des proprietes emergentes inatendues et/ou indesirables, venant d'un probleme local qui a des consequences globales.

\end{itemize} 

L'analyse systejme est base sur une modelisation qui ne 
La démarche d'étude conduit logiqument a un ordonnancement des différents diagrammes.

\subsection{Système cible}


\subsection{Angles d'Analyse}

\subsection{}


\section{Identification du périmètre systèmique}


\subsection{Environement et systèmes externes}
\subsection{Parties prenantes}
\subsection{Diagramme de contexte} 



\section{Analyse et modélisation opérationnelle}
\subsection{Expression des besoins}
\subsection{Diagramme de cycle de vie}

\subsection{Diagramme de scénario opérationnel}


\section{Analyse et modélisation fonctionnelle}



\subsection{Exigences fonctionnelles}

\clearpage


\subsection{Diagramme de scénario fonctionnel}

\clearpage


\subsection{}

\clearpage


\section{Analyse et modélisation organique}



\subsection{Exigences techniques}

\clearpage


\subsection{Diagramme de scénario organique}

\clearpage


\subsection{}

\clearpage


\

\section{Annexes}


\end{document}